\documentclass[a4paper,12pt]{article}

\usepackage[english]{babel}
\usepackage[utf8x]{inputenc}
\usepackage{amsmath}
\usepackage{graphicx}
\usepackage{amssymb}
\usepackage{hyperref}
\usepackage{amsmath}
\usepackage{prettyref}
\usepackage{verbatim}
\usepackage[square,sort,comma,numbers]{natbib}
\usepackage[nohyphen]{underscore}
\usepackage[capitalise]{cleveref}
\usepackage[colorinlistoftodos]{todonotes}

\title{Finding Bot Accounts on Twitter via Tweets}
\author{David Tomassi, Somdutta Bose, Saja Alayadhi}

\begin{document}
\maketitle

\begin{abstract}
The goal is to detect bot accounts on Twitter via the granularity of tweets and their contents. Recently, we have seen the phenomena of ``Fake News'' and people's ability to buy bot accounts to respond to tweets and make them seem more popular. The significance of the project is to be able to see which accounts on Twitter are actually human. The data set we used is the MIB Dataset (http://mib.projects.iit.cnr.it/dataset.html) which has a genuine and bot account information and their respective tweets. We tokenized tweets and did an embedding (GloVe and Word2vec) of the tokens and then appended them together to be the input to our model. We investigate the effectiveness of different approaches using a single layer feed-forward neural network, multiple layer feed-forward neural network and Long short-term memory (LSTM). The results we present, show how accurate and precise the model is at determining if a Twitter account is a bot by the tweets.
\end{abstract}

\section{Introduction}
\label{sec:introduction}

Spam emails, advertisements, and communications in social media are a
common part of the internet and using these services. Detecting these
spam outlets has been a rich area of research
\citep{Cormack:2008:ESF:1454707.1454708,
DBLP:journals/corr/cs-CL-0009009, Androutsopoulos2006LearningTF,
Bickel:2006:DSF:2976456.2976477, Bratko:2006:SFU:1248547.1248644,
Solan:inproceedings, Cresci:2017:PSS:3041021.3055135, fameforsale2015,
INUWADUTSE2018496, FM2793, 8424744}.  With "Fake" news being spread
throughout social media \citep{NBERw25223} it has become imperative
that new methods and techniques be developed to address this
phenomenon. The focus on the social media platform Twitter has become
emergent \citep{8424744, FM2793, INUWADUTSE2018496,
Cresci:2017:PSS:3041021.3055135, fameforsale2015} as bot accounts
inflating the popularity of certian ideas has become common. 


\section{Related Work}
\label{sec:relatedwork}

In \cite{DeepNeuralNetworks}, the authors propose a deep neural network based on contextual long short-term memory (LSTM) architecture. They exploit both the content and metadata to detect bots at the tweet level. They extracted contextual features from user metadata and fed as auxiliary input  to LSTM deep nets processing the tweet text. They also proposed a technique based on synthetic minority oversampling to generate large dataset, suitable for deep nets training, from a minimum amount of labeled data. They demonstrated that with their architecture they achieved high classification accuracy (AUC \textgreater $96$\%) in separating bots from humans. When they applied the same architecture to account-level bot bot detection, they achieved nearly perfect classification accuracy (AUC \textgreater $99$\%). In \cite{DetectingSpamAccounts}, the authors propose a novel approach for distinguishing spam from no-spam social media post. They optimized a set of features independent of historical tweets. These tweets were available only for a short time on Twitter. Account features related to users were taken into account. They observed that an average automated spam account posted at-least  $12$ tweets a day at well defined periods. Their approach achieved a significant improvement on performance when compared to exiting spam detection techniques.

\section{Methods}
\label{sec:methods}

\subsection{Dataset} The dataset we will be using for our training and
evaluation will be the My Information Bubble (MIB) dataset
\citep{Cresci:2017:PSS:3041021.3055135}. It is a collection of genuine
and spam tweets from bot accounts. There are thousands of accounts
with millions of tweets. We will split the dataset by accounts with a
90/10 split for training and validation data.

\subsection{Pre-Processing} We will pre-process the tweets in order to
normalize their format and reduce vocabulary size when we create an
embedding.

\subsubsection{Parsing} To parse the tweets, we employ the
text-processing tool Ekphrasis
\citep{baziotis-pelekis-doulkeridis:2017:SemEval2} that will performn
tokenization, normalization and word segmentation. Using Ekphrasis
allows for a uniform format for the tweets and abstracts away
usernames and emojis which will decress the vocabulary size that will
be used in the embedding.

\subsubsection{Embedding} Since our vocabulary size will be large as
the breadth of language, slang, and domain specific words is diverse.
Using a one-hot-encoding, would increase the dimensonality of the
input by a large margin. To overcome this obstricle, we will be use a
Word2Vec embedding \citep{Mikolov:2013:DRW:2999792.2999959} which will
decrease the dimensonality to 100 to represent one word. In particular
we will train a Word2Vec
model\footnote{https://radimrehurek.com/gensim/models/word2vec.html}
on our tweets and use the embedding to encode them for input to the
model.

\subsection{Pruning}

\subsection{Feed-Forward Neural Network}

\subsection{Long Short-Term Memory}

\subsection{Bidirectional Encoder Representations from Transformers}


\section{Results}
\label{sec:results}

\subsection{Feed-Forward Neural Network}

\subsubsection{Single Hidden Layer} The single hidden layer
feed-forward neural network was able to achieve an accuracy of 87.3\%.

\subsubsection{Multiple Hidden Layers} The multiple hidden layer
feed-forward neural network was able to achieve an accuracy of 93.5\%.

\subsection{Long Short-Term Memory}


\section{Discussion}
\label{sec:discussion}

\subsection{Feed-Forward Neural Network} Both feed-forward networks
achieved good accuracy on the validation set. The multiple hidden
layer architecure was able to perform better than its single layer
counter part. This could be because of the deeper layers learning more
complex relationships between the tokens in the tweet. As the single
hidden layer was learning patterns of spam tweets.

\subsection{Long Short-Term Memory }
The assumption was that LSTM would achieved a batter accuracy, but it did not. In our problem the single hidden layer feed forward network achieved a better accuracy. However, we cannot tell if the difference in the accuracy because of the model choice or the embedding approach since we used different approach for each model. One of the future work that can be done here is to do the same LSTM model with word embedding that are trained in our data instead of using the pre trained Glove. 
In order to try to increase the accuracy, different hyper-parameters for the model were examined but none of which achieved better result. Moreover, we modified the model architecture to include convolution layer with max pooling but the accuracy did not improve. Another suggestion for future work is to increase the number of neurons in the LSTM layer but that would need of course more computational power. 


\section{Conclusions}
This project aimed to classify a twitter account as a spam or not spam by using only a tweet text. We successfully got a good accuracy in two different approaches for the word embedding and tried to develop four models. The best accuracy obtained was $93.5$\%, from training using multiple hidden layer feed-forward neural network with Word2Vec embedding. We also observe that we got better results with Word2Vec embedding. However, for future work can involve LSTM with Word2Vec embedding and using GloVe with single-layer feed-forward neural network and multiple-layer feed-forward neural network.
\label{sec:conclusions}

%\section{References}
\label{sec:references}

\subsection{Email Spam Detection}

\subsection{Twitter Spam Detection}



\section{Author Contributions}
\label{sec:author-contributions}

\subsection{Saja Alayadhi}

\subsection{Somdutta Bose}

\subsection{David Tomassi}
\begin{itemize}
    \item Requesting Access to MIB Dataset
    \item Word2Vec Embedding
    \item Single Hidden Layered Feed-Forward Neural Network
    \item Multiple Hidden Layered Feed-Forward Neural Network
\end{itemize}


\bibliographystyle{splncs04nat}
\bibliography{main}

\end{document}