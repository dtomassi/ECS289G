\section{Introduction}
\label{sec:introduction}

Spam emails, advertisements, and communications in social media are a
common part of the internet and using these services. Detecting these
spam outlets has been a rich area of research
\citep{Cormack:2008:ESF:1454707.1454708,
DBLP:journals/corr/cs-CL-0009009, Androutsopoulos2006LearningTF,
Bickel:2006:DSF:2976456.2976477, Bratko:2006:SFU:1248547.1248644,
Solan:inproceedings, Cresci:2017:PSS:3041021.3055135, fameforsale2015,
INUWADUTSE2018496, FM2793, 8424744}.  With "Fake" news being spread
throughout social media \citep{NBERw25223} it has become imperative
that new methods and techniques be developed to address this
phenomenon. The focus on the social media platform Twitter has become
emergent \citep{8424744, FM2793, INUWADUTSE2018496,
Cresci:2017:PSS:3041021.3055135, fameforsale2015} as bot accounts
inflating the popularity of certian ideas has become common. Current
approaches to detect bot accounts use hand-engineered features which
can be troublesome to retrieve.

In this paper we introduce several different approaches to solve the
problem of detecting a bot account on Twitter by just a tweet. We use
different deep learning approaches using a feed-forward neural
network, an Long Short-Term Memory (LSTM) based architecture, and
Bidirectional Encoder Representations from Transformers (BERT) with
the goal of identifying a bot account by just a tweet. Since only a
tweet is required for classification, users do not have to be
encumbered with gathering a multitude of different data ranging from
number of followers, age of the account, and etc. This makes the
approach easier to use than previous methods.

The rest of the paper will be as follows: \cref{sec:methods} will
detail our approach, the results of our experiments will be described
\cref{sec:results}, we have a discussion on the results in \cref{sec:discussion}, we conclude in \cref{sec:conclusions}, in
\cref{sec:references} we discuss previous work, and in
\cref{sec:author-contributions} we detail each author's contributions.
